\documentclass[12pt,a4paper]{amsart}
% ukazi za delo s slovenscino -- izberi kodiranje, ki ti ustreza
\usepackage[slovene]{babel}
\usepackage[utf8]{inputenc}
\usepackage[T1]{fontenc}
\usepackage{multirow}
\usepackage{amsmath,amssymb,amsfonts}
\usepackage{url}
\usepackage[normalem]{ulem}
\usepackage[dvipsnames,usenames]{color}
\usepackage{csquotes}
\usepackage{caption}
\usepackage{lipsum}
\usepackage{hyperref}
\usepackage{tikz}
\usepackage{listings}
\usepackage{xcolor}
\usepackage{graphicx}
\usepackage{subcaption}


\lstset{
    breaklines=true,    
    breakatwhitespace=false,
    postbreak=\space,   
    tabsize=2,    
    basicstyle=\small\ttfamily\bfseries,
    commentstyle=\color{green!50!black},
    keywordstyle=\color{blue},
    numberstyle=\tiny\color{gray},
    numbers=left
}
\usetikzlibrary{graphs}
\usetikzlibrary{graphs.standard}

\makeatletter
\renewcommand\section{\@startsection{section}{1}%
  \z@{.5\linespacing\@plus.7\linespacing}{.5\linespacing}%
  {\normalfont\scshape\large\centering}}
\renewcommand\subsection{\@startsection{subsection}{2}%
  \z@{.5\linespacing\@plus.7\linespacing}{.5\linespacing}%
  {\normalfont\scshape}}
\renewcommand\subsubsection{\@startsection{subsubsection}{3}%
  \z@{.5\linespacing\@plus.7\linespacing}{-.5em}%
  {\normalfont\itshape}}
\makeatother

% ne spreminjaj podatkov, ki vplivajo na obliko strani
\textwidth 15cm
\textheight 24cm
\oddsidemargin.5cm
\evensidemargin.5cm
\topmargin-5mm
\addtolength{\footskip}{10pt}
\pagestyle{plain}
\overfullrule=15pt % oznaci predlogo vrstico


% ukazi za matematicna okolja
\theoremstyle{plain} % tekst napisan pokoncno
\newtheorem{definicija}{Definicija}[section]
\newtheorem{primer}[definicija]{Primer}
\newtheorem{definition}{Definicija}[section]

\renewcommand\endprimer{\hfill$\diamondsuit$}

% za stevilske mnozice uporabi naslednje simbole
\newcommand{\R}{\mathbb R}
\newcommand{\N}{\mathbb N}
\newcommand{\Z}{\mathbb Z}
\newcommand{\C}{\mathbb C}
\newcommand{\Q}{\mathbb Q}

% ukaz za slovarsko geslo
\newlength{\odstavek}
\setlength{\odstavek}{\parindent}
\newcommand{\geslo}[2]{\noindent\textbf{#1}\hspace*{3mm}\hangindent=\parindent\hangafter=1 #2}

% naslednje ukaze ustrezno popravi
\newcommand{\program}{Finančna matematika} % ime studijskega programa: Matematika/Finančna matematika
\newcommand{\imeavtorja}{Tian Lipovšek, Klara Gruden} % ime avtorja
\newcommand{\imementorja}{doc. dr. Janoš Vidali} % akademski naziv in ime mentorja
\newcommand{\imesomentorja}{prof. dr. Riste Škrekovski}
\newcommand{\naslovdela}{Lokalna k-matrična dimenzija grafov}
\newcommand{\letnica}{2024} %letnica 


%%%%%%%%%%%%%%%%%%%%%%%%%%%%%%%%%%%%%%%%%%%%%%%%%%%%%%%%%%%%%%%%%%%%%%%%%%%%%%%%%%%%%%%%%%
\begin{document}

\thispagestyle{empty}
\noindent{\large
UNIVERZA V LJUBLJANI\\[1mm]
FAKULTETA ZA MATEMATIKO IN FIZIKO\\[5mm]
\program\ }
\vfill

\begin{center}{\large
\imeavtorja\\[2mm]
{\bf \naslovdela}\\[10mm]
Skupinski projekt\\[2mm]
Poročilo\\[1cm]
Mentorja: \imementorja, \\ \imesomentorja\\[2mm]}
\end{center}
\vfill

\noindent{\large
Ljubljana, januar \letnica}
\pagebreak

%%%%%%%%%%%%%%%%%%%%%%%%%%%%%%%%%%%%%%%%%%%%%%%%%%%%%%%%%%%%%%%%%
\section{Navodilo naloge}
Following the paper [10], implement an ILP model for this invariant, and then write separate
small programs in Sage to answer each of following questions by exhaustive search.
\begin{enumerate}
    \item Determine $ldim_k(G)$ for paths, cycles, complete graphs, bipartite complete graphs, hypercubes and some 
    other interesting classes of graphs and try to guess the possible formulas based on the computations.
    \item Try to determine graphs G which satisfy $ldimk(G) = dimk(G)$ for a given $k$ for $k=1,2,3,...$.
    For small graphs, apply a systematic search; for larger ones, apply some stochastic search.
\end{enumerate}
%%%%%%%%%%%%%%%%%%%%%%%%%%%%%%%%%%%%%%%%%%%%%%%%%%%%%%%%%%%%%%%%%
\section{Način reševanja problema}

\bigskip
Za reševanje opisanega problema sva uporabila okolje Sage (SageMath) znotraj spletne platforme CoCalc. 
V nadaljevanju bova opisala in dodala kodo le od nekaterih funkcij, celotna koda s komentarji
se nahaja na GitHub repozitorju, prilagava tudi \href{https://github.com/TianLipovsek/ProjektFP_Local-k-Metric-Dimension}{povezavo}.


\bigskip
%%%%%%%%%%%%%%%%%%%%%%%%%%%%%%%%%%%%%%%%%%%%%%%%%%%%%%%%%%%%%%%%%
\section{Definicije}
Za lažje razumevanje najinega problema, si najprej poglejmo par definicij, ki sva jih uporabljala v sklopu projektne naloge.
\begin{definition}
        Vozlišče $s$ reši par vozlišč $x$, $y$ v grafu $G$, če $d(s,x) \neq d(s,y)$.
    \end{definition}

    \begin{definition}
        Rešujoča množica grafa $G$ je množica $S$, za katero velja, da za vsaki dve 
        vozlišči $x$ in $y$ v $V(G)$ obstaja vozlišče $s \in S$, ki reši par vozlišč $x$,$y$.
    \end{definition}

    \begin{definition}
        Metrična dimenzija povezanega grafa $G$, označena z $dim(G)$, je definirana kot velikost najmanjše 
        množice $S \subseteq V(G)$, ki razlikuje vse pare vozlišč v $G$.
    \end{definition}     

    \begin{definition}
        Lokalna rešujoča množica grafa $G$ je množica $S$, za katero velja, da za vsaki sosednji 
        vozlišči $x$ in $y$ v $V(G)$ obstaja vozlišče $s \in S$, ki reši par vozlišč $x$,$y$.
    \end{definition}

    \begin{definition}
        K-metrična dimenzija, označimo jo s $dim_k(G)$ je metrična dimenzija, 
        pri kateri hočemo, da za vsak par vozlišč obstaja vsaj k vozlišč $s \in S$, ki rešijo par vozlišč $x$,$y$.
    \end{definition}   
    
    \begin{definition}
        Lokalna metrična dimenzija, označena z $ldim(G)$, predstavlja moč najmanjše 
        lokalno rešujoče množice grafa $G$.
    \end{definition} 

\bigskip


%%%%%%%%%%%%%%%%%%%%%%%%%%%%%%%%%%%%%%%%%%%%%%%%%%%%
\section{Reševanje problema in ugotovitve}
Najprej sva napisala funkcijo CLP ki izračuna lokalno
k-metrično dimenzijo grafa. Funkcija sprejme graf G in število k, ter vrne vrednost lokalne k-metrične dimenzije
ob upoštevanju potrebnih pogojev.
\bigskip 
\begin{small}
    \begin{lstlisting}
    import itertools

   def CLP_local_k_metric_dim(g, k_value):
    p = MixedIntegerLinearProgram(maximization=False)
    x = p.new_variable(binary=True)
    for vi in g.vertices():
        neighbors = g.neighbors(vi)
        expr = sum(x[vi] + x[vj] for vj in neighbors) + x[vi]
        p.add_constraint(expr >= k_value)
    p.set_objective(sum(x[vi] for vi in g))
    optimal_solution = p.solve()
    values_for_x = p.get_values(x)
    #return int(optimal_solution), values_for_x
    return optimal_solution
    \end{lstlisting}
\end{small}
\bigskip


\bigskip

%%%%%%%%%%%%%%%%%%%
\subsection{Prvi del naloge}
s pomočjo zgornje funkcije sva tako določila lokalne k-metrične dimenzije za različno velike grafe različnih skupin.

\begin{enumerate}
\item Grafi poti: \\
najprej sva si pogledala grafe poti. S $n$ je označeno število vozlišč grafov poti.  
 
 
\begin{table}[h]
    \begin{minipage}{.33\linewidth}
        \centering
        \begin{tabular}{cc}
            \textbf{n} & \textbf{ldim$_1$} \\
            \hline
            2 & 1 \\
            3 & 1 \\
            4 & 2 \\
            5 & 2 \\
            6 & 2 \\
            7 & 3 \\
            8 & 3 \\
            9 & 3 \\
            10 & 4 \\
        \end{tabular}
        \label{tab:tabela1}
    \end{minipage}%
    \begin{minipage}{.33\linewidth}
        \centering
        \begin{tabular}{cc}
            \textbf{n} & \textbf{ldim$_2$} \\
            \hline
            2 & 2 \\
            3 & 2 \\
            4 & 3 \\
            5 & 3 \\
            6 & 4 \\
            7 & 4 \\
            8 & 5 \\
            9 & 5 \\
            10 & 6 \\
        \end{tabular}
        \label{tab:tabela2}
    \end{minipage}%
    \begin{minipage}{.33\linewidth}
        \centering
        \begin{tabular}{cc}
            \textbf{n} & \textbf{ldim$_3$} \\
            \hline
            2 & 2 \\
            3 & 3 \\
            4 & 4 \\
            5 & 5 \\
            6 & 6 \\
            7 & 7 \\
            8 & 8 \\
            9 & 9 \\
            10 & 10 \\
        \end{tabular}
        \label{tab:tabela3}
    \end{minipage}
\end{table}

tabele prikazujejo vrednosti, ki sva jih dobila za lokalne k-metrične dimenzije grafov poti pri \\ $k = 1, 2, 3$. na podlagi teh podatkov sva oblikovala približne formule za izračun $ldim_k$. \\

\begin{itemize}
    \item za $k=1$: $ldim_k= \lceil \frac{n}{3} \rceil$ 
    \item za $k = 2$: $ ldim_k= \lceil \frac{n+1}{2} \rceil $
    \item za $k= 3$: $ ldim_k= n $
\end{itemize}

\newpage

\item Grafi ciklov: \\
  $n$ prav tako prikazuje število vozlišč grafov. Pri tej družini grafov so vrednosti, ki jih dobiva za $ldim_k$ zelo podobne kot pri grafih poti z razliko, da tukaj dobimo vrednosti za k = 1,2,3,4,5

\begin{table}[h]
    \begin{minipage}{.33\linewidth}
        \centering
        \begin{tabular}{cc}
            \textbf{n} & \textbf{ldim$_1$} \\
            \hline
            3 & 1 \\
            4 & 2 \\
            5 & 2 \\
            6 & 2 \\
            7 & 3 \\
            8 & 3 \\
            9 & 3 \\
            10 & 4 \\
        \end{tabular}
        \label{tab:tabela1}
    \end{minipage}%
    \begin{minipage}{.33\linewidth}
        \centering
        \begin{tabular}{cc}
            \textbf{n} & \textbf{ldim$_2$} \\
            \hline
            3 & 2 \\
            4 & 2 \\
            5 & 3 \\
            6 & 3 \\
            7 & 4 \\
            8 & 4 \\
            9 & 5 \\
            10 & 5 \\
        \end{tabular}
        \label{tab:tabela2}
    \end{minipage}%
    \begin{minipage}{.33\linewidth}
        \centering
        \begin{tabular}{cc}
            \textbf{n} & \textbf{ldim$_k$} \\
            \hline
            3 & 3 \\
            4 & 4 \\
            5 & 5 \\
            6 & 6 \\
            7 & 7 \\
            8 & 8 \\
            9 & 9 \\
            10 & 10 \\
        \end{tabular}
        \label{tab:tabela3}
    \end{minipage}
\end{table}

Na podlagi zgoraj prikazanih vrednosti sva oblikovala sledeče formule:
\begin{itemize}
    \item za $k=1$: $ldim_k= \lceil \frac{n}{3} \rceil$ 
    \item za $k = 2$: $ ldim_k= \lceil \frac{n}{2} \rceil $
    \item za $k= 3,4,5$: $ ldim_k= n $
\end{itemize}

\item Polni grafi: \\
$n$ = število vozlišč grafa.

\begin{table}[h]
    \begin{center}
        \begin{tabular}{|c|c|c|c|c|c|}
            \hline
            $n$ & $ldim_1$ & $ldim_2$ & $ldim_4$ & $ldim_6$ & $ldim_9$ \\
            \hline
            
            
            3 & 1 & 2 & 3 & 3 & 3 \\
            4 & 1 & 2 & 4 & 4 & 4 \\
            5 & 1 & 2 & 4 & 5 & 5 \\
            6 & 1 & 2 & 4 & 6 & 6 \\
            7 & 1 & 2 & 4 & 6 & 7 \\
            8 & 1 & 2 & 4 & 6 & 8 \\
            9 & 1 & 2 & 4 & 6 & 9 \\
            10 & 1 & 2 & 4 & 6 & 9 \\
            11 & 1 & 2 & 4 & 6 & 9 \\
            12 & 1 & 2 & 4 & 6 & 9 \\
            \hline
        \end{tabular}
    \end{center}
    \caption{Primer tabele z $ldim_k$ vrednostmi}
    \label{tab:tabela_ldim_k}
\end{table}

V tej skupini grafov pa sva iz dobljenih vrednosti prikazanih tudi v zgornji tabeli sklepala da velja za iračun $ldim_k$ splošna formula :

\begin{equation}
ldim_k =
\begin{cases}
  n, & \text{če } n \geq k, \\
  k, & \text{če } n < k. \nonumber
\end{cases}
\end{equation}

kjer je maksimalen $k$ za posamezen $n$ lahko $k = 2n -1$ .

\newpage

\item Polni dvodelni grafi:

Polne grafe v $sagemath$ zrišemo kot : $graphs.CompleteBipartiteGraph(n, m)$, kjer je potem skupno število vozlišč $n+m$ . 
Iz dobljenih vrednosti pa sklepava da velja spodnja formula. 

\begin{equation}
ldim_k =
\begin{cases}
  2k, & \text{če } n,m \geq 2k, \\
  m, & \text{če } n = k-1, m\geq k,\\
  n, & \text{če } m = k-1, n\geq k,\\
  \min{m,n}, & \text{če } (n \in [k,2k) \land  m\geq k) \lor (m \in [k,2k) \land  n\geq k) , \\
  n+m & \text{če } (n,m \in [\lceil \frac{k}{2} \rceil,k) ) \lor
  (m = \lceil \frac{k}{2} \rceil \land  n\geq \lceil \frac{k}{2} \rceil)  \lor
  (n = \lceil \frac{k}{2} \rceil \land  m\geq \lceil \frac{k}{2} \rceil) . \nonumber
   
   
  
\end{cases}
\end{equation}

\item Hiperkocke: \\

hiperkocke v $sagamath$ zrišemo kot :
$graphs.CubeGraph(m)$. Število  oglišč hiperkocke $m $
je tako $n= 2^m $

\begin{table}[h]
    \begin{center}
        \begin{tabular}{|c|c|c|c|c|c|}
            \hline
            $m$ & 2 & 3 & 4 & 5 & 6 \\
            \hline
            
            
            $ldim_1$ & 2 & 2 & 4 & 7 & 12 \\
            $ldim_2$ & 2 & 4 & 7 & 10 & 16 \\
            $ldim_3$ & 4 & 4 & 8 & 16 & 24 \\
            $ldim_4$ & 4 & 8 & 8 & 16 & 32 \\
            $ldim_5$ & 4 & 8 & 16 & 16 & 32 \\
            $ldim_6$ & / & 8 & 16 & 32 & 32 \\
            $ldim_7$ & / & 8 & 16 & 32 & 64 \\
            $ldim_8$ & / & / & 16 & 32 & 64 \\
            $ldim_9$ & / & / & 16 & 32 & 64 \\
            
            \hline
        \end{tabular}
    \end{center}
    
    
\end{table}

iz nekaj dobljenih vrednosti sklepava, da je formula za izračun lokalne k-metrične dimenzije :

\begin{equation}
ldim_k =
\begin{cases}
  2^m , & \text{če } k > m, \\
  2^{m-1} , & \text{če } k = m,\\
  ?, & \text{če } k < m, \\
  /, & \text{če } k >2 m +1 . \nonumber
\end{cases}
\end{equation}



\item Povezani cikel v hiperkocki: \\
v tej družini grafov je zelo podobno kot pri hiperkockah. število vozlišč je toktat $n= 2^m \cdot m$. tako sklepava, da je na podlagi pridobljenih vrednosti furmula za lokalni k-metrično dimenzijo:

\begin{table}[h]
    \begin{center}
        \begin{tabular}{|c|c|c|c|c|c|}
            \hline
            $m$ & 1 & 2 & 3 & $4$ & $5$ \\
            \hline
            
            
            $ldim_1$ & 1 & 3 & 6 & 16 & ? \\
            $ldim_2$ & 2 & 4 & 11 & ? & ? \\
            $ldim_3$ & 2 & 8 & 16 & ? & ? \\
            $ldim_4$ & 2 & 8 & 24 & 32 & ? \\
            $ldim_5$ & 2 & 8 & 24 & 64 & 80 \\
            $ldim_6$ & / & / & 24 & 64 & 160 \\
            $ldim_7$ & / & / & 24 & 64 & 160 \\
            $ldim_8$ & / & / & / & 64 & 160 \\
            $ldim_9$ & / & / & / & 64 & 160 \\
            
            \hline
        \end{tabular}
    \end{center}
    
    
\end{table}



\begin{equation}
ldim_k =
\begin{cases}
  2^m \cdot m, & \text{če } k > m, \\
  2^{m-1} \cdot m, & \text{če } k = m,\\
  ?, & \text{če } k < m, \\
  /, & \text{če } k >2 m +1 . \nonumber
\end{cases}
\end{equation}


\end{enumerate}  



\bigskip


%%%%%%%%%%%%%%%%%%%
\subsection{Drugi del naloge}

Za začetek sva potrebovala CLP za izračun k-metrične dimenzije grafa, ki je podoben tistemu, 
ki sva ga napisala za lokalno k-metrično dimenzijo.  
Nato sva napisala funkcijo poimenovano $naloga2\_1$, ki sprejme graf in vrednost k. Ta funkcija poišče lokalno k-metrično ter k-metrično dimenzijo na grafu, ter preveri ujemanja, na koncu pa vrne slovar, v katerem so ključi slovarja tiste vrednosti, pri katerih se dimenziji ujemata.

Za tem sva potrebovala še funkcijo $testiranje\_naloga2\_1(m, n, max\_k)$, 
ki bo s pomočjo funkcije $naloga2\_1$ na grafih s številom vozlišč med m in n poiskala 
tiste, pri katerih se ujemata k-ti dimenziji ter take grafe tudi izriše.

Najprej sva se osredotočila na posebne vrste grafov, kot so poti, cikli ter polni grafi. 
Pri poteh sva ugotovila, da se ujemajo za majhne grafe do 4 vozlišča, saj se pri 
večjem število vozlišč lokalna k-metrična dimenzija veča precej hitreje kot navadna 
k-metrična dimenzija ter tako ne pride več do ujemanj. Nato sva pogledala cikle, pri 
katerih sva našla ujemanja za grafe do 6 vozlišč za k do 1 in 2, pri čemer je bilo za 
cikel velikosti 5 in 6 ujemanje tudi za k=4. Pri ciklih pa ni bilo ujemanj za cikel 
velikosti tri, kar je poln graf na treh vozliščih. Prav pri polnih grafih sva namreč 
ugotovila, da je prišlo do ujemanj le za graf velikosti 2, torej le za pot 
dolžine 2, kar pa smo si pogledali že v prejšnji točki, za vse ostale polne grafe, 
ki so torej na vozliščih več kot 3, ni nobenih ujemanj lokalne k-metrične ter 
k-metrične dimenzije.

Pri testiranju s funkcijo s pomočjo vgrajene funkcije nauty geng sva se osredotočila na iskanje ujemanj za grafe različnih velikosti za določen k. Najprej sva si pogledala ujemanja za k enak dva in našla poleg takih grafov, ki sva jih omenila že zgoraj (poti), ujemanja pri 7 vozliščih na dveh grafih. To sta:

\bigskip

\begin{figure}
    \begin{minipage}{0.5\textwidth}
      \centering
      \includegraphics[width=\linewidth]{naloga2_primer1.png}
      \caption{Graf 1}
      \label{fig:slika1}
    \end{minipage}%
    \begin{minipage}{0.5\textwidth}
      \centering
      \includegraphics[width=\linewidth]{naloga2_primer2.png}
      \caption{Graf 2}
      \label{fig:slika2}
    \end{minipage}
  \end{figure}

\bigskip
Potem sva nadaljevala z iskanjem in ugotovila, da se pri večjih grafih, ki si med sabo niso precej podobni najde kar nekaj grafov, vendar pa se s številom vozlišč tudi število grafov precej veča. 
Pogledala sva si še ujemanje dimenzij za k=3 in našla ujemanja na grafu s petimi vozlišči za dva taka grafa, nato pa se je število že nekoliko bolj povečalo. Zato sva prišla do zaključka, da se res pri večjih grafih najde veliko število različnih grafov, pri katerih se dimenzije ujemajo, vendar je izjemno težko opaziti podobnosti med vsemi grafi. 

\bigskip


%%%%%%%%%%%%%%%%%%%%%%%%%%%%%%%%%%%%%%%%%%%%%%%%%%%%%%%%%%%%%%%%%

%%%%%%%%%%%%%%%%%%%%%%%%%%%%%%%%%%%%%%%%%%%%%%%%%%%%%


%%%%%%%%%%%%%%%%%%%%%%%%%%%%%%%%%%%%%%%%%%%%%%%%%%%%%
\bigskip
Za pomoč pri nalogi sva uporabila spodnjo literaturo.
\begin{thebibliography}{99}
    \bibitem{peterin2023resolving}
    I. Peterin, J. Sedlar, R. Škrekovski, I. G. Yero,
    \emph{Resolving vertices of graphs with differences},
    (2023) arXiv preprint arXiv:2309.00922.
    \end{thebibliography}

\end{document}